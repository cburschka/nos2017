\begin{frame}{}
	\begin{center}
		\scalebox{1.5}{\Huge \textcolor{emorot}{Subtraktion}}
		
		\vspace{1em}
		
		{\large \textcolor{goetheblau}{Minus-Rechnen}}
	\end{center}
\end{frame}


\begin{frame}{Subtraktion}
	\begin{center}
		\Huge

		\only<+>{
			\scalebox{1.5}{\Huge \textcolor{emorot}{Selbsttest}}
		}		
		\only<+>{$\textcolor{goetheblau}{7}-\textcolor{emorot}{3}=\underline{\phantom4}$}
		\only<+>{$\textcolor{goetheblau}{7}-\textcolor{emorot}{3}=\textcolor{orange}{\underline{4}}$}
		
		\only<+>{
			\scalebox{1.5}{\Huge \textcolor{emorot}{Und noch einer \ldots}}
		}		
		\only<+>{$\textcolor{goetheblau}{123}-\textcolor{emorot}{45}=\underline{\phantom{78}}$}
		\only<+>{
			$\textcolor{goetheblau}{123}-\textcolor{emorot}{45}=\textcolor{orange}{\underline{78}}$
			\begin{tikzpicture}[overlay, remember picture]
				\node[at=(current page.south east), anchor=south east, xshift=-2em, yshift=2em]  {\Large \color{sandgrau} \#NotMyPisa};
			\end{tikzpicture}
		}
	\end{center}
\end{frame}

\begin{frame}{Unäre Zahlen -- römisch für Arme}
	\begin{tikzpicture}[remember picture, overlay]
	\node[at=(current page.north), yshift=-3em, anchor=north, draw, fill=sandgrau, minimum width=2\textwidth, minimum height=2.5em] {
		\large
		Wir stellen die \textcolor{goetheblau}{Zahl $x$} durch  \textcolor{emorot}{$x$ Einsen} dar.
	};
	\end{tikzpicture}


	\begin{center}
		\begin{tikzpicture}[
			node distance=0,
			dnum/.style={align=right, text width=2em},
			unum/.style={align=right, text width=8em}
		]
			\node (a0) {};
			\foreach \i in {1, ..., 5} {
				\node[dnum, below=of a\intcalcDec\i] (a\i) {$\textcolor{goetheblau}{\i}_\text{Dez}$};
				\node[right=of a\i] (b\i) {$=$};
				\node[unum, right=of b\i] (c\i) {$\textcolor{emorot}{\ttunum{\i}}_\text{Unär}$};
	
				\node[dnum, right=2em of c\i] (d\i) {$\textcolor{goetheblau}{\intcalcAdd5\i}_\text{Dez}$};
				\node[right=of d\i] (e\i) {$=$};
				\node[unum, right=of e\i] (f\i) {$\textcolor{emorot}{\ttunum{\intcalcAdd5\i}}_\text{Unär}$};
			}
		\end{tikzpicture}
	\end{center}
\end{frame}

\begin{frame}{Wie beschreiben wir unäre Subtraktion?}
	\vspace{4.5em}

	\begin{center}
		\scalebox{1.3}{
			\begin{tikzpicture}[
				number/.style={text width=3.8em, align=right, anchor=north west, inner sep=0.1em, minimum height=1.5em, orange},
				header/.style={number, font={\bf}}
			]
				\def\sx{4em}
				\def\sy{1.5em}
	
				\node[fill=sandgrau, anchor=north west, minimum width=7*\sx, minimum height=\sy, inner sep=0] at (0,0)  {};
				\node[fill=sandgrau, anchor=north west, minimum width=\sx, minimum height=11em, inner sep=0] at (0,0)  {};
			
				\foreach \x in {1, ..., 5} {
					\node[header, goetheblau] at (\x*\sx, 0) {$\mathbf{\unum{\x}}$};
					\node[header, emorot] at (0, -\x*\sy) {$\mathbf{\unum{\x}}$};
	
					\node[header, goetheblau] at (6*\sx, 0) {$\cdots$};
					\node[header, emorot, minimum height=2em] at (0, -6*\sy) {$\vdots$\ \ \phantom.};
	
					\node[number, fill=sandgrau] at (6*\sx, -\x*\sy) {$\cdots$};
					\node[number, fill=sandgrau, minimum height=2em] at (\x*\sx, -6*\sy) {$\vdots$};
	
					
					\foreach \y in {1, ..., 5} {
						\ifthenelse{\equal{\intcalcCmp\x\y}{-1}}{
							\node[number, fill=sandgrau] at (\x*\sx, -\y*\sy) {};
						}{
							\node[number] at (\x*\sx, -\y*\sy) {
								\ifthenelse{\equal\x\y}{\textcolor{emorot}{\only<2>{0}}} 
								{\ttunum{\intcalcSub\x\y}}
							};
						}
					}
				}
				
				\node[header, fill=sandgrau, minimum height=2em] at (6*\sx, -6*\sy) {$\ddots$};
				
				\node[goetheblau, draw, minimum width=28em, minimum height=11em, anchor=north west, onslide={<4->white, fill=white, opacity=0.8}] (tabborder) {};
			\end{tikzpicture}
		}
	\end{center}

	\begin{tikzpicture}[remember picture, overlay]
		\node[at=(current page.north), yshift=-3em, anchor=north, draw, fill=sandgrau, minimum width=2\textwidth, minimum height=2.5em] {
			\Huge
			$\textcolor{goetheblau}{a} - \textcolor{emorot}{b} = \textcolor{orange}{c}$
		};
		
		\only<4->{
			\node[draw, radius=0.5em, rounded corners=1em, emorot, fill=white, opacity=0.8, inner sep=2em] at (current page) {\phantom{\Huge Nicht praktikabel für große Zahlen!}};
			
			\node[emorot, inner sep=2em] at (current page) {\Huge Unpraktisch für große Zahlen!};
		}
	\end{tikzpicture}
\end{frame}

\newcommand{\removeDigit}[1]{%
	\only<-\intcalcDec{#1}>{1}%
	\only<#1>{\textcolor{emorot}{\underline 1}}%
}

\begin{frame}{Wir brauchen einen Algorithmus}
	\begin{tikzpicture}[remember picture, overlay]
	\node[at=(current page.north), yshift=-3em, anchor=north, draw, fill=sandgrau, minimum width=2\textwidth] {
		\Huge
		$\textcolor{goetheblau}{a} - \textcolor{emorot}{b} = \textcolor{orange}{c}$
	};
	\end{tikzpicture}
	
	\begin{columns}[t]
		\begin{column}{0.7\textwidth}
			\Large
			\setbeamertemplate{enumerate items}[default]
			\begin{enumerate}
				\item
					\tikzanchor{step1}%
					\textbf{Wenn} \textcolor{emorot}{$b$} keine Ziffer mehr hat,\\
					\tikzanchor{step1a}
					\textbf{dann} \textcolor{orange}{\textbf{antworte} $c \gets a$} und \textbf{halte}
					
					\vspace{1em}

				
				\item
					\tikzanchor{step2}%
					\texttt{Entferne} letzte Ziffer von \textcolor{goetheblau}{a}\\
					\texttt{Entferne} letzte eine Ziffer von \textcolor{emorot}{b}

					\vspace{1em}

				
				\item 
					\tikzanchor{step3}%
					\textbf{Springe} zu \textcolor{goetheblau}{1.}
			\end{enumerate}
		\end{column}%
		
		\begin{column}{0.25\textwidth}
			\Huge

			$\textcolor{goetheblau}{a}$: \hfill 1\removeDigit6\removeDigit3\\
			$\textcolor{emorot}{b}$: \hfill \removeDigit6\removeDigit3\\
			$\textcolor{orange}{c}$: \hfill \only<-8>{?}\only<9>{\textcolor{emorot}{1}}
			
			\uncover<9>{\begin{center}
					Halt, Stop!
			\end{center}}
			
		\end{column}
	\end{columns}
	
	\begin{tikzpicture}[
		overlay, remember picture,
		prog counter/.style={emorot, <-, ultra thick}
	]
		\path[prog counter, onslide={<2,5,8>draw}] ($(step1)+(-2.5em,0)$) to ++(-2em, 0);
		\path[prog counter, onslide={<9>draw}]     ($(step1a)+(-2.5em,0)$) to ++(-2em, 0);		
		\path[prog counter, onslide={<3,6>draw}]   ($(step2)+(-2.5em,0)$) to ++(-2em, 0);
		\path[prog counter, onslide={<4,7>draw}]   ($(step3)+(-2.5em,0)$) to ++(-2em, 0);
	
	\end{tikzpicture}
	
\end{frame}

