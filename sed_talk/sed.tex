\documentclass[aspectratio=169,usenames,dvipsnames]{beamer}

\usepackage{color, colortbl}

\usepackage[USenglish]{babel}
\usepackage{amsmath,amsfonts,amssymb}
\usepackage{graphicx}
\usepackage{pdfpages}

\usepackage{latexsym}
\usepackage{xspace}

%TODO: Cyriax should install packages...
%\usepackage{siunitx}
%\sisetup{locale = DE}

\usepackage{xstring}
\usepackage{tikz}
\usepackage{pgf}


\usetikzlibrary{calc}   
\usetikzlibrary{positioning}
\usetikzlibrary{automata}

\usepackage{soul}

% Make it different to distinguish it from Manus stuff
\usetheme{Berlin}
\usecolortheme{dolphin}

\usepackage{cancel}
\usepackage{ulem}




\begin{document}
\title{
      Mein supertoller \texttt{sed} Vortrag
   }
\author{
	Jonathan Cyriax Brast\\
	Manuel Penschuck \\
} 
\date{10. Juni 2017}

{
\setbeamertemplate{footline}{} 
\begin{frame}
  \titlepage
\end{frame}
}
\addtocounter{framenumber}{-1}

\newcommand{\bsl}{{\ttfamily \symbol{`\\}}}
\newcommand{\cA}[1]{{\color{NavyBlue}#1}}
\newcommand{\cB}[1]{{\color{YellowOrange}#1}}
\newcommand{\cC}[1]{{\color{Violet}#1}}
\newcommand{\cD}[1]{{\color{Red}#1}}
\newcommand{\cE}[1]{{\color{OliveGreen}#1}}
\newcommand{\cF}[1]{{\color{CadetBlue}#1}}
\newcommand{\cG}[1]{{\color{Gray}#1}}
\newcommand{\sedunder}[2]{$\underset{\text{#1}}{\text{#2}}$}
%\frame{\frametitle{Outline}\tableofcontents}

%%%%%%%%%%%%%
\begin{frame}{\texttt{sed} - H\"a was?}
	{
	\Huge
	\only<1,3->{\texttt{sed}} % besteht aus den drei buchstaben
	\only<2>{\texttt{s e d}}\quad
	\only<4-5>{\only<5>{\xout}{''Suchen und Ersetzen in Dateien''}}
	\only<6->{''\textbf{S}tream \textbf{Ed}itor''}\\
	}
	\pause\pause\pause\pause\pause\pause
	{\tt sed} ist ein echtes ausf\"uhrbares Programm \\\pause
	{\tt sed} kann Text(-dateien) automatisch \"andern\\\pause
	{\tt sed} sucht Dinge und ersetzt sie durch Zeuch\\\pause
	{\Large \quad\quad ergo folgt deshalb daraus:\\\pause}
	{\tt sed} ist viel cooler als Turing Maschinen\\\pause
	{\hspace{22em} $\square$ q.e.d. \texttt{\#isso}}
\end{frame}

\begin{frame}{Womit arbeitet {\tt sed}}
	\begin{tabular}{ll}

		\textit{\cA{h}\cB{a}\cC{ll}\cD{o}} &
		ein \cA{h}, ein \cB{a}, zwei \cC{l} und ein \cD{o}\\&
		\textit{''hallo''}\vspace{.2em}
		\\\pause
		\textit{\cA{h}\cB{.}\cC{ll}\cD{o}} &
		ein \cA{h},  ein \cB{beliebiges Zeichen},  zwei \cC{l} und ein \cD{o}\\&
		\textit{''hallo'' \quad  ''hello'' \quad  ''h4llo''}
		\\\pause
		\textit{\cA{.}\cD{.}\cC{.}}&
		\cA{irgendein Zeichen},  \cD{noch ein Zeichen} und \cC{noch ein Zeichen}\\&
		\textit{''ABC'' \quad ''123'' \quad ''{\tt sed}''}
		\\\pause
		\textit{\cA{ha}\cB{l*}\cD{o}} &
		erst \cA{ha}, dann \cB{beliebig viele l} und dann ein \cD{o}\\&
		\textit{''hallo'' \quad ''halo'' \quad ''hao'' \quad  ''hallllllllo''}
		\\\pause
		\textit{\cA{ha}\cD{.*}} &
		\cA{ha} und dann \cD{irgendwas} \\&
		\textit{''hallo'' \quad ''hartgekochter'' \quad ''hamster'' \quad ''haha'' \quad ''ha''}
	\end{tabular}
\end{frame}

\begin{frame}{{\tt sed} - Grundlegende Kommandos}
	Suchen und Ersetzen (alles andere is unwichtig):\\\pause
	\begin{tabular}{ll}
		{\tt s/ / /}
		&
		{\tt s} wie ''suchen und ersetzen''\\\pause
		{\tt s/ \cA{Dinge} / \cB{Zeuch} /}
		&
		'' \cA{Dinge} '' $\rightarrow$ \pause '' \cB{Zeuch} ''\\\pause
		{\tt s/ \cA{.*} / \cB{Zeuch} /}
		&
		'' \cA{irgendwas} '' $\rightarrow$ \pause '' \cB{Zeuch} ''\\\pause
		{\tt s/ \cA{\sedunder{\bsl1}{(Dinge)}} \cB{\sedunder{\bsl2}{(Zeuch)}} / \cB{\bsl2} \cA{\bsl1} /}
		&
		'' \cA{Dinge} \cB{Zeuch} '' $\rightarrow$ \pause '' \cB{Zeuch} \cA{Dinge} ''
	\end{tabular}\pause

	\bigskip
	Sonstige Befehle:\\\pause
	\begin{tabular}{ll}
		{\tt :\cA{irgendwo}} & Eine Sprungmarke\\\pause
		{\tt t \cA{irgendwo}} & Springe zur Sprungmarke\\\pause
		{\tt p } & Drucke aus, was du grade denkst
	\end{tabular}

\end{frame}


\begin{frame}{ Zust\"ande und \"Uberg\"ange }
	Zustands\"ubergang:\\
	\hspace{12em}
	''\cC{ZUSTAND}''\\\pause
	\bigskip	
	\hspace{15em}
	\texttt{s/\visible<2->{\cC{ZUSTAND-ALT}}/%
	\visible<3->{\cC{ZUSTAND-NEU}}/}\\
	\pause\pause
	\bigskip
	Entscheidung:\\
	\hspace{12em}
	''\cC{ZUSTAND}:\cB{X}''\\\pause
	\bigskip
	\hspace{15em}
	\texttt{s/\visible<5->{\cC{ZUSTAND-ALT}:}\visible<6->{\cB{X}}/\visible<8->{\cC{ZUSTAND-1}:\cB{X}}/}\\
	\hspace{15em}
	\texttt{s/\visible<5->{\cC{ZUSTAND-ALT}:}\visible<7->{\cB{U}}/\visible<9->{\cC{ZUSTAND-2}:\cB{U}}/}
\end{frame}

\begin{frame}{BandKopfApparat}
	\hspace{8em}
	''\cA{00100100001001\cD{@}\cB{0}10100101010}''\\\pause
	\bigskip
	e lesen und \"a schreiben:\\
	\hspace{15em}
	''\only<5->\cA{Hallo W}\only<3->\cD{@}\only<4->\cB{e}\only<5->\cA{lt}''
	$\rightarrow$
	''\only<5->\cA{Hallo W}\only<3->\cD{@}\only<4->\cB{\"a}\only<5->\cA{lt}''
	\\
	\hspace{15em}
	\texttt{%
		s/%
		\visible<5->{\cA{\sedunder{\bsl1}{(.*)}}}%
		\visible<3->{\cD{@}}\visible<4->{\cB{e}}%
		\visible<5->{\cA{\sedunder{\bsl2}{(.*)}}}/%
	\visible<5->{\cA{\bsl1}}%
	\visible<3->{\cD{@}}\visible<4->{\cB{\"a}}%
	\visible<5->{\cA{\bsl2}}/}
	\\\pause\pause\pause\pause

	Nach links gehen:\\
	\hspace{15em}
	''\only<9->\cA{Hallo }\only<8->\cB{W}\only<7->\cD{@}\only<9->\cA{elt}''
	$\rightarrow$
	''\only<9->\cA{Hallo }\only<7->\cD{@}\only<8->\cB{W}\only<9->\cA{elt}''
	\\
	\hspace{15em}
	\texttt{%
		s/\visible<9->{\cA{\sedunder{\bsl1}{(.*)}}}%
		\visible<8->{\cB{\sedunder{\bsl2}{(.)}}}\visible<7->{\cD{@}}%
		\visible<9->{\cA{\sedunder{\bsl3}{(.*)}}}/%
	\visible<9->{\cA{\bsl1}}\visible<7->{\cD{@}}\visible<8->{\cB{\bsl2}}\visible<9->{\cA{\bsl3}}/}
\end{frame}


\begin{frame}{{\tt sed} Turingmaschinenzustands\"uberg\"ange}
	\hspace{8em}
	''\cC{ZUSTAND}:\cA{001001001011\cE{1}\cD{@}\cB{0}001010010011}''\\\pause
	\bigskip
	\texttt{s/%
	\visible<4->{\cC{ZUSTAND-ALT}:}%
	\visible<10->{\cA{\sedunder{\bsl1}{(.*)}}}%
	\visible<9->{\cE{\sedunder{\bsl2}{(.)}}}%
	\visible<6->{\cD{@}}%
	\visible<7->{\cB{X}}%
	\visible<10->{\cA{\sedunder{\bsl3}{(.*)}}}/%
	\visible<4->{\cC{ZUSTAND-NEU}:}%
	\visible<10->{\cA{\bsl1}}%
	\visible<6->{\cD{@}}%
	\visible<9->{\cE{\bsl2}}%
	\visible<7->{\cB{Y}}%
	\visible<10->{\cA{\bsl3}}%
	/}\\
	\pause
	\bigskip

	%TODO: Manu tikz Bild
	1. Gehe von \cC{ZUSTAND-ALT} in den \cC{ZUSTAND-NEU}\\\pause\pause
	2. Lese ein \cB{X} und schreibe ein \cB{Y}\\\pause\pause\pause
	3. Gehe nach links oder rechts (vetausche \cD{@} und \cE{1}):
	
\end{frame}

\begin{frame}{ Die {\tt sed} Turingmaschine}
	
	\tt
	\visible<3->{s/\cA{(.*)}/\cC{STARTZUSTAND}:\cD{@}\cA{\bsl1}/}\\
	\visible<2->{:\cB{nochmal}}~\\
	\visible<2->{p}~\\
	\textit{Zustands\"ubergang}\visible<2->{ ; t \cB{nochmal}}\\
	\textit{Zustands\"ubergang}\visible<2->{ ; t \cB{nochmal}}\\
	\textit{Zustands\"ubergang}\visible<2->{ ; t \cB{nochmal}}\\
	\visible<4->{s/\cD{@}//}\\
	\visible<5->{s/\cC{(.*)}:\cA{(.*)}/\cA{\bsl2}/}

\end{frame}

\begin{frame}{colortest}

	{ \color{BrickRed} X}
	{ \color{red} X}
	{ \color{Maroon} X}
	{ \color{Brown} X}
	{ \color{Sepia} X}
	\\
	{ \color{PineGreen} X}
	{ \color{OliveGreen} X}
	{ \color{ForestGreen} X}
	{ \color{LimeGreen} X}
	{ \color{RoyalBlue} X}
	{ \color{NavyBlue} X}
	\\
	{ \color{BurntOrange} X}
	{ \color{YellowOrange} X}
	{ \color{Yellow} X}
	{ \color{Goldenrod} X}
	\\
	\cA{cA}\cB{cB}\cC{cC}\cD{cD}\cE{cE}
	\\
	{ \color{YellowOrange} X}
	{ \color{NavyBlue} X}
	%{ \color{LimeGreen} X}
	{ \color{red} X}
	{ \color{Violet} X}
	{ \color{OliveGreen} X}
	{ \color{CadetBlue} X}
	{ \color{Gray} X}
\end{frame}


\end{document}

