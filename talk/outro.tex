\begin{frame}{Was haben wir gesehen?}
	\uncover<4->{\Large \textbf{Turing-Vollständigkeit}}	
	\begin{itemize}
		\item \textcolor{goetheblau}{sed} kann eine \textcolor{emorot}{Turing Maschine} ausführen
		\pause
		\item Eine \textcolor{emorot}{Turing Maschine} kann \textcolor{goetheblau}{sed} ausführen
		\pause
		\item[$\Rightarrow$] Beide sind gleich mächtig!
	\end{itemize}
		
	\vspace{1em}
	\pause
\end{frame}

\begin{frame}{Was ist nicht Turing-Vollständig}
	\begin{itemize}
		\item
			\textbf{Turing Maschine ohne Band} (= sed ohne Sprünge)\\
		\pause	
			Reguläre Sprachen
			\vspace{1em}

		\pause			
		\item
			\textbf{Turing Maschinen mit Keller-Speicher} (LiFo)\\
		\pause
			Kontext-freie Sprachen
			\vspace{1em}			
			
		\pause
		\item
			Datenbeschreibungssprachen:
			XML, HTML, JSON, YAML, CSV, Markup
			\vspace{1em}
						
	\end{itemize}
\end{frame}

\begin{frame}[fragile]{Achja \ldots}
\begin{center}
\ldots die Turing Maschine am Anfang haben wir direkt in \LaTeX{} berechnet.

\vspace{1em}

\pause
\textcolor{goetheblau}{Dieser Foliensatz ist also auch Turing vollständig.}
\end{center}
%
%\pause	
%\begin{lstlisting}
%\def\tmMoveRightHelp#1@#2@#3#4\relax{{%
%	\ifx#3\relax
%		\xdef\tmNewBand{#1#2@B@}%
%	\else
%		\xdef\tmNewBand{#1#2@#3@#4}%
%	\fi
%}}
%
%\newcommand{\tmMoveRight}{%
%	\def\tmNewBand{}%
%	\expandafter\tmMoveRightHelp\tmBand\relax\relax\relax\relax\relax%
%	\xdef\tmBand{\tmNewBand}
%}
%\end{lstlisting}
\end{frame}


\setbeamertemplate{footline}{} 
\goethccBgTitel
\begin{frame}{}
	\titlepage
	\begin{tikzpicture}[overlay]
	\node[anchor=south east, xshift=-0.08\textwidth, at=(current page.south east)] {
		\parbox{0.3\textwidth}{
		\begin{center}
			\scalebox{1.5}{\qrcode{http://nos.manuel.jetzt/}}
			
			\vspace{1em}
			
			\small\textcolor{goetheblau}{\url{http://nos.manuel.jetzt}}
		\end{center}
	}};
	\end{tikzpicture}
\end{frame}
