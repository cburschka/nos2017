%%%%%%%%%%%% Turing Machine State Diagram
\def\tmKeyBasename#1#2.#3\relax{\xdef#1{#2}}

\newcommand{\tmDrawStates}[1]{
	% set \tmHeadChar
	\tmCurrentChar
	
	
    \def\tmstate##1##2##3##4{
        % 1 key
        % 2 label
        % tikz options
        % tikz position

        \node[
            tmState,
            onif={<\equal{##1}{\tmStateCurrent}>tmStateActive},
            ##3
        ] (##1) ##4 {##2};
    }

    \def\tmstart##1{
        \path[tmTrans]  ($(##1.west)-(1.5em, 0)$) to ++(1.5em, 0);
    }

    \def\tmtrans##1##2##3##4##5{{
        % 1 current state
        % 2 next state
        % 3 instruction
        % 4 format path
        % 5 format label

		% Disable the newline comannd (only for this scope, though)
		\renewcommand{\\}{}
		
		\tmKeyBasename\tmDrawSourceBase##1.\relax		
		\tmKeyBasename\tmDrawTargetBase##2.\relax		

		% First find out if this is an active edge
        \def\tmDrawTransActiveEdge{}
        {
			% Parameter ##3 contains a list of \trans commands;
			% that's all we are interested in. Gather the infos
			% via the local version of \trans!
		    \def\trans####1####2####3{%
	    		\ifthenelse{\equal{\tmStateCurrent}{\tmDrawSourceBase} \AND \equal{\tmHeadChar}{####1}}{%
	    			\gdef\tmDrawTransActiveEdge{tmTransActive}%
	    		}{}%
		    }%
			##3
        }
        
		% Build the label and store it in the following macro        
        \def\tmDrawTransLabel{}
	    {
	        % Parameter ##3 contains a list of \trans commands;
			% that's all we are interested in. Gather the infos
			% via the local version of \trans!
            \def\trans####1####2####3{%
            	% Human readable direction
            	\ifthenelse{\equal{####3}{L}}{%
            		\def\tmDrawTransDir{$\Leftarrow$}%
            	}{\ifthenelse{\equal{####3}{R}}{%
            		\def\tmDrawTransDir{$\Rightarrow$}%
            	}{%
	            	\def\tmDrawTransDir{$\Downarrow$}%
	            }}%

				% Make active transition label red, remainder black
				\ifthenelse{\equal{\tmStateCurrent}{\tmDrawSourceBase} \AND \equal{\tmHeadChar}{####1}}{
					\edef\transLabelColor{emorot}
				}{
					\edef\transLabelColor{black}
				}

				% Short-Cut             	
            	\def\transLabel{%
            		\noexpand\textcolor{\transLabelColor}{%
	            		\noexpand\texttt{####1} zu \noexpand\texttt{####2}, \tmDrawTransDir%
	            	}%
            	}%

				\xdef\tmDrawTransLabel{
					\unexpanded\expandafter{\tmDrawTransLabel}%
					\tmDrawTransLabelNewLine % Empty in first iteration, otherwise \\
					\transLabel
		    	}%
		    	
			    % Seperate several transition with newline
		    	\def\tmDrawTransLabelNewLine{\noexpand\\}
            }%

	        \def\tmDrawTransLabelNewLine{}
            ##3
        }
        

        \path[tmTrans, \tmDrawTransActiveEdge, ##4] 
	        (##1) to 	% source
	        node[align=center, ##5] {\tmDrawTransLabel} % label
	        (##2) % target
	    ;
    }}

	\begin{tikzpicture}[
		node distance = 4em and 15em,
		on grid
	]
		#1
    \end{tikzpicture}
}



%%%%%%%%%%%% Turing Machine Band Diagram and Band Helpers
\def\tmCurrentCharHelp#1@#2@#3\relax{\xdef\tmHeadChar{#2}}%
\newcommand{\tmCurrentChar}{\expandafter\tmCurrentCharHelp\tmBand\relax}


\newcounter{tmBandCnt}
\newcounter{tmBandCntHead}

\newcommand{\tmDrawBand}{
	\def\decreasedCounter##1{\numexpr\value{##1} - 1\relax}
	
	\def\tmbandCellCaption##1{%
		\ifthenelse{\equal{##1}{B}}{$\mathbb B$}{##1}%
	}	
	
	\def\turingBandHelp##1##2##3##4\relax{%
		\ifx##1\relax{}\else{%
			\stepcounter{tmBandCnt}
			\ifthenelse{\equal{##1}{@}}{%
				\node[tmCellActive, onif={<\equal{##2}{B}>tmCellBlank}] (tmbandcell\thetmBandCnt) at ($(tmbandcell0) + (\thetmBandCnt*\cellSize,0)$) {\tmbandCellCaption##2};
				\setcounter{tmBandCntHead}{\thetmBandCnt}%
				\turingBandHelp##4\relax\relax\relax\relax%		
			}{%
				\node[tmCell, onif={<\equal{##1}{B}>tmCellBlank}] (tmbandcell\thetmBandCnt) at ($(tmbandcell0) + (\thetmBandCnt*\cellSize,0)$) {\tmbandCellCaption##1};
				\turingBandHelp##2##3##4\relax\relax\relax%
		}%
	}
	\fi%
	}
	
	
	
	\begin{tikzpicture}[remember picture]
		\node[tmCell] (tmbandcell0) {};
	
		\setcounter{tmBandCnt}{0}
		\setcounter{tmBandCnt}{0}
		\expandafter\turingBandHelp\tmBand\relax\relax\relax\relax
	
		\ifthenelse{\equal{\thetmBandCntHead}{0}}{}{
			\node[tmHeadMarker, at=(tmbandcell\thetmBandCntHead)] (tmheadmarker) {};				
		}
	\end{tikzpicture}
}

\newcommand{\tmDrawGrid}[2]{
	\foreach \i in {#1, ..., #2} {
		\path[draw] ($(tmbandcell0.north west)+(\i*\cellSize, 0)$) 
		to ($(tmbandcell0.south west)+(\i*\cellSize, 0)$);
	}
	
	\path[draw] ($(tmbandcell0.north west)+(#1*\cellSize, 0)$) 
	to ($(tmbandcell0.north west)+(#2*\cellSize, 0)$);
	
	\path[draw] ($(tmbandcell0.south west)+(#1*\cellSize, 0)$) 
	to ($(tmbandcell0.south west)+(#2*\cellSize, 0)$);
}




%%%%%%%%%%%% Turing Machine Execution
\def\tmMoveLeftHelp#1#2#3#4#5\relax{{%
	\ifx#1@%
		% Move left before start: Insert Blank Symbol
		\xdef\tmNewBand{@B@#2#4#5}%
	\else%
		\ifx#4\relax%
			% This is not possible in an input containing @
			ILLEGAL INPUT%
		\else
			\ifx#2@%
				\xdef\tmNewBand{\tmNewBand @#1@#3#5}%
			\else%
				\xdef\tmNewBand{\tmNewBand#1}%
				\tmMoveLeftHelp#2#3#4#5\relax\relax%
			\fi%
		\fi%
	\fi
}}

\def\tmMoveRightHelp#1@#2@#3#4\relax{{%
	\ifx#3\relax
		\xdef\tmNewBand{#1#2@B@}%
	\else
		\xdef\tmNewBand{#1#2@#3@#4}%
	\fi
}}

\newcommand{\tmMoveLeft}{%
	\def\tmNewBand{}%
	\expandafter\tmMoveLeftHelp\tmBand\relax\relax\relax\relax\relax%
	\xdef\tmBand{\tmNewBand}
}

\newcommand{\tmMoveRight}{%
	\def\tmNewBand{}%
	\expandafter\tmMoveRightHelp\tmBand\relax\relax\relax\relax\relax%
	\xdef\tmBand{\tmNewBand}
}




\def\tmWriteCharHelp#1@#2@#3\relax{\xdef\tmBand{#1@\tmCharToWrite@#3}}%
\newcommand{\tmWriteChar}[1]{%
	\def\tmCharToWrite{#1}%
	\expandafter\tmWriteCharHelp\tmBand\relax%
}

\def\tmExecTrans#1#2#3#4#5\relax{
	\if\tmHeadChar#1
		\tmWriteChar#2
		\if#3L
			\tmMoveLeft
		\else\if#3R
			\tmMoveRight
		\fi\fi
		\csname#4\endcsname
	\else
		\if\relax\detokenize{#5}\relax\else
			\tmExecTrans#5\relax
		\fi
	\fi
}

\def\tmShowBandHelp#1@#2@#3\relax{%
	\texttt{#1\underline{#2}#3}%
}

\newcommand{\tmShowBand}{
	\expandafter\tmShowBandHelp\tmBand\relax	
}


%%%%%%%%%%%%% TM Parser
\newcommand{\tmSetup}[2]{{
	% This macro will interpret a TM specification 
	% as defined the three operations:
	%  \tmstate{X}{}{}{} where X is the state's key
	%  \tmstart{X} where X is the start state's key
	%  \tmtrans{X}{Y}{Z}{}{} where 
	%     X is the source state's key
	%     Y is the target state's key
	%     Z is a transition list consisting of 
	%     \trans{A}{B}{C} commands where
	%		A is the read character
	%       B the replacement
	%       C the movement {L, R, H}
	
	% The setup process creates a number of macrocs
	% with a tmImpl#1 prefix. Each state X gets two
	% macros tmImpl#1TransX which contains a list of
	% transitions (each transition is defined by 
	% exactly four tokens) and tmImpl#1X which is
	% "callable" transition.
		
	\def\tmCurrentName{#1}
	
	\newcommand{\tmstate}[4]{%
		\expandafter\xdef\csname tmImpl\tmCurrentName Trans##1\endcsname{}
		\expandafter\xdef\csname tmImpl\tmCurrentName ##1\endcsname{}
	}
	
	\newcommand{\tmstart}[1]{
		\expandafter\xdef\csname tmImpl\tmCurrentName Start\endcsname{##1}
	}

	\newcommand{\trans}[3]{
		\xdef\tmTransAccum{\tmTransAccum##1##2##3{\tmTransTargetKey}}
	}
	\newcommand{\tmtrans}[5]{
		\tmKeyBasename\tmImplSourceBase##1.\relax		
		\tmKeyBasename\tmImplTargetBase##2.\relax
		
		\xdef\tmTransTargetKey{tmImpl\tmCurrentName\tmImplTargetBase}
		\def\tmTransAccum{}
		
		
		##3
		
		\expandafter\xdef\csname tmImpl\tmCurrentName Trans\tmImplSourceBase\endcsname{%
			\csname tmImpl\tmCurrentName Trans\tmImplSourceBase\endcsname%
			\tmTransAccum
		}

		\expandafter\xdef\csname tmImpl\tmCurrentName\tmImplSourceBase\endcsname{%
			\noexpand\tmCurrentChar%
			\noexpand\csname tmImpl\tmCurrentName StateCallback\endcsname{\tmImplSourceBase}%
			\noexpand\tmExecTrans\csname tmImpl\tmCurrentName Trans\tmImplSourceBase\endcsname\noexpand\relax
		}
	}

	% tmTrans Definitions may include newlines; redefine command to prevent output
	\renewcommand{\\}{}	
	#2
}}


\def\tmSetInitialBand#1#2\relax{\xdef\tmBand{@#1@#2}}
\newcommand{\tmExecute}[3]{
	\tmSetInitialBand#2\relax
	\expandafter\gdef\csname tmImpl#1StateCallback\endcsname{#3}
	\csname tmImpl#1\csname tmImpl#1Start\endcsname\endcsname
}
